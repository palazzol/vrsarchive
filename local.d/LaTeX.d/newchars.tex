\documentstyle{report}
\setlength{\topmargin}{.25in}
\setlength{\oddsidemargin}{0in}  %  needs negative value
\setlength{\headheight}{.75in}
\setlength{\headsep}{.5in}
%\setlength{\footheight}{1in} 	%dont need them for qms, but
%\setlength{\footskip}{1.5in}	%for imagen, yes.
\setlength{\textheight}{8in}
\setlength{\textwidth}{5.5in} 
\setlength{\marginparwidth}{0in}
\setlength{\marginparsep}{0in}
\setlength{\marginparpush}{0in}
\renewcommand{\baselinestretch}{1.5}  % for kind of double spacing

\begin{document}
\newcommand{\circlems}{\circle{1.0}}
\newcommand{\circlescs}{\circle{2.0}}
\newcommand{\circlecs}{\circle*{1.0}}
\setlength{\unitlength}{1mm}

\section*{A Variety Of Characters For Latex}

{\raggedleft Sunil Podar (podar@sbcs)}

\medskip
This document contains my experiments and their results at 
discovering new characters that can be plotted in the pic environment of 
LaTeX. My main motivation has been for plotting points on graphs but
certainly the characters listed below and their construction is not limited
to just graphs.
These are in addition to the
standard \verb+\circle+ and \verb+\circle*+ characters. Since for other 
characters one
doesn't have explicit control over the size, one has to play around with the
fonttypes, sizes, etc. and quite interestingly one can virtually get
anything, or close to anything, desired. Somewhere hidden in THE manual is the
important bit of information that the macros \verb+\makebox & \framebox+ are
available outside the pic environment \& so is \verb+\shortstack+.

\medskip
The essential idea is using a \verb+\makebox(0,0){.........}+ so that the 
\verb+\put+
statement will put the specified box "right on the dot". The actual arguements
for the box size might vary depending on document pointsize, unitlength,
etc., particularly when it comes to centering circles in boxes.
Using such a setup of boxes of size (0,0), one has a whole new range
of mathematical symbols available; of course, the useful ones are the 
symmetric symbols. 
Some of the other possible candidates among the math symbols:
\verb+\times, \ast, \star, \oplus, \Diamond, \bigoplus,+ etc.
See pages 49-51 of THE manual.
One interesting feature I discoverd is that
one can use framebox of appropiate size around any object or no object at all.
Also it seems not all the fonts can be used for characters since the
characters are not necessarily centered in their own little ``font-boxes'' (I
don't know what the technical term is, perhaps raster ); e.g. I tried the
character {\tt +} in the tt font and it is a bit off-center; same story with
the {\tt *}. Obviously the {\it italic} font is out of the question.

\medskip
The following are with all with unitlength = 1mm.
Also they are all set in 10pt size, so other sizes will produce
correspondingly bigger characters, I think. 
The characters seem to not get centered properly, but that is because they
are getting lined up with the bottom of the line. I have tried them all in
actual graphs and they all appear exactly on the coordinates specified.

\newpage
\medskip
\begin{tabular}{@{}cl}
\underline{\bf character} & \underline{\bf LaTeX command.}\\[2ex]
{\makebox(0,0){$\bigotimes$}}	
		& \verb+{\makebox(0,0){$\bigotimes$}}+ \\
{\makebox(0,0){\LARGE $\otimes$}} 
		& \verb+{\makebox(0,0){\LARGE $\otimes$}}+ \\
{\makebox(0,0){\large $\otimes$}} 
		& \verb+{\makebox(0,0){\large $\otimes$}}+\\
{\makebox(0,0){$\textstyle \otimes$}} 
		& \verb+{\makebox(0,0){$\textstyle \otimes$}}+ \\
{\makebox(0,0){$\scriptstyle \otimes$}}	
		& \verb+{\makebox(0,0){$\scriptstyle \otimes$}}+ \\
{\makebox(0,0){$\scriptscriptstyle \otimes$}} 
		& \verb+{\makebox(0,0){$\scriptscriptstyle \otimes$}}+ \\
{\makebox(0,0){\framebox(1.5,1.5){}}} 
		& \verb+{\makebox(0,0){\framebox(1.5,1.5){}}}+ \\
{\makebox(0,0){\framebox(2,2){$+$}}} 
		& \verb={\makebox(0,0){\framebox(2,2){$+$}}}= \\
{\makebox(0,0){\framebox(2,2)[lb]{\put(1,1){\circlecs}}}} 
	& \verb+{\makebox(0,0){\framebox(2,2)[lb]{\put(1,1){\circlecs}}}}+ \\
{\makebox(0,0){\framebox(1.5,1.5){.}}} 
		& \verb+{\makebox(0,0){\framebox(1.5,1.5){.}}}+ \\
{\makebox(0,0){\framebox(1.5,1.5){\Large .}}} 
		& \verb+{\makebox(0,0){\framebox(1.5,1.5){\Large .}}}+ \\
{\makebox(0,0){$\bigodot$}}	
		& \verb+{\makebox(0,0){$\bigodot$}}+ \\
{\makebox(0,0){\LARGE $\odot$}} 
		& \verb+{\makebox(0,0){\LARGE $\odot$}}+ \\
{\makebox(0,0){\large $\odot$}} 
		& \verb+{\makebox(0,0){\large $\odot$}}+\\
{\makebox(0,0){$\textstyle \odot$}} 
		& \verb+{\makebox(0,0){$\textstyle \odot$}}+ \\
{\makebox(0,0){$\scriptstyle \odot$}}	
		& \verb+{\makebox(0,0){$\scriptstyle \odot$}}+ \\
{\makebox(0,0){$\scriptscriptstyle \odot$}} 
		& \verb+{\makebox(0,0){$\scriptscriptstyle \odot$}}+ \\
{\makebox(0,0){\LARGE $\heartsuit$}}
		& \verb+{\makebox(0,0){\LARGE $\heartsuit$}}+ \\[1ex]
{\makebox(0,0){$\textstyle \oplus$}} 
		& \verb+{\makebox(0,0){$\textstyle \oplus$}}+ \\
{\makebox(0,0){$\textstyle \star$}} 
		& \verb+{\makebox(0,0){$\textstyle \star$}}+ \\
{\makebox(0,0){\rule{2mm}{2mm}}}
		& \verb+{\makebox(0,0){\rule{2mm}{2mm}}}+ \\
&
\end{tabular}

\end{document}
