% AMS-TEX VERSION 0.999999 - FOR USE WITH TEX VERSION  1.0
% COPYRIGHT (C) 1983 BY AMERICAN MATHEMATICAL SOCIETY


% NOTE 1, NOTE 2, ... REFER TO NOTES IN THE FILE AMSTEX.DOC


% SPECIAL CATCODES

\catcode`\@=13                                                         % NOTE 1
\def@{\errmessage{AmS-TeX error: \string@ has no current use
     (use \string\@\space for printed \string@ symbol)}}
\catcode`\@=11                                                         % NOTE 2
\def\@{\char'100 }
\catcode`\~=13                                                         % NOTE 3


% AMSTEX ERROR MESSAGES

\def\err@AmS#1{\errmessage{AmS-TeX error: #1}}                         % NOTE 4


% SOME BASIC CONTROL SEQUENCES USED IN OTHER DEFINITIONS

\def\eat@AmS#1{}

\long\def\comp@AmS#1#2{\def\@AmS{#1}\def\@@AmS{#2}\ifx
   \@AmS\@@AmS\def\cresult@AmS{T}\else\def\cresult@AmS{F}\fi}          % NOTE 5

\def\in@AmS#1#2{\def\intest@AmS##1#1##2{\comp@AmS##2\end@AmS\if T\cresult@AmS
   \def\cresult@AmS{F}\def\in@@AmS{}\else
   \def\cresult@AmS{T}\def\in@@AmS####1\end@AmS{}\fi\in@@AmS}%
   \def\cresult@AmS{F}\intest@AmS#2#1\end@AmS}                         % NOTE 6


% BASIC MECHANICSMS TO ALLOW USER TO MAKE DEFINITIONS

\let\relax@AmS=\relax                                                  % NOTE 7

% CHANGES IN plain WHERE THERE IS \relax, THAT MUST NOW BE \relax@AmS

\def\magstep#1{\ifcase#1 \@m\or 1200\or 1440\or 1728\or 2074\or 2488\fi
     \relax@AmS}

\def\iterate{\body\let\next\iterate \else\let\next\relax@AmS\fi \next}

\def\enskip{\hskip.5em\relax@AmS}

\def\strut{\relax@AmS\ifmmode\copy\strutbox\else\unhcopy\strutbox\fi}

\let\+=\relax@AmS
\def\sett@b{\ifx\next\+\let\next\relax@AmS
    \def\next{\afterassignment\s@tt@b\let\next}%
  \else\let\next\s@tcols\fi\next}
\def\s@tt@b{\let\next\relax@AmS\us@false\m@ketabbox}

\def\smash{\relax@AmS % \relax@AmS, in case this comes first in \halign
  \ifmmode\def\next{\mathpalette\mathsm@sh}\else\let\next\makesm@sh
  \fi\next}

% (END OF CHANGES TO plain)

\def\define#1{\expandafter\ifx\csname\expandafter\eat@AmS\string#1\endcsname
   \relax@AmS\def\dresult@AmS{\def#1}\else
   \err@AmS{\string#1\space is already defined}\def
      \dresult@AmS{\def\garbage@AmS}\fi\dresult@AmS}                   % NOTE 8

\let\redefine=\def

\def\predefine#1#2{\let#1=#2}


% MACROS FOR DEFICIENT KEYBOARDS

\chardef\plus=`+
\chardef\equal=`=
\chardef\less=`<
\chardef\more=`>


% MACROS FOR HANDLING TEXT

\let\ic@AmS=\/
\def\/{\unskip\ic@AmS}

\def\Space@AmS.{\futurelet\Space@AmS\relax@AmS}
\Space@AmS. %                                             % NOTE 10 (no NOTE 9)

\def~{\unskip\futurelet\tok@AmS\s@AmS}
\def\s@AmS{\ifx\tok@AmS\Space@AmS\def\next@AmS{}\else
        \def\next@AmS{\ }\fi\penalty 9999 \next@AmS}                  % NOTE 11

\def\period{\unskip.\spacefactor3000 { }}


\def\srdr@AmS{\thinspace}                                             % NOTE 12
\def\drsr@AmS{\kern .02778em }
\def\sldl@AmS{\kern .02778em}
\def\dlsl@AmS{\thinspace}

\def\lqtest@AmS#1{\comp@AmS{#1}`\if T\cresult@AmS\else\comp@AmS{#1}\lq\fi}

                                                                      % NOTE 13

\def\qspace#1{\unskip
  \lqtest@AmS{#1}\let\fresult@AmS=\cresult@AmS\if T\cresult@AmS
     \def\qspace@AmS{\ifx\tok@AmS\Space@AmS\def\next@AmS{\dlsl@AmS`}\else
       \def\next@AmS{\qspace@@AmS}\fi\next@AmS}\else
     \def\qspace@AmS{\ifx\tok@AmS\Space@AmS\def\next@AmS{\drsr@AmS'}\else
       \def\next@AmS{\qspace@@AmS}\fi\next@AmS}\fi
    \futurelet\tok@AmS\qspace@AmS}                                    % NOTE 14

\def\qspace@@AmS{\futurelet\tok@AmS\qspace@@@AmS}

\def\qspace@@@AmS{\if T\fresult@AmS  \ifx\tok@AmS`\sldl@AmS`\else
       \ifx\tok@AmS\lq\sldl@AmS`\else \dlsl@AmS`\fi \fi
                         \else  \ifx\tok@AmS'\srdr@AmS'\else
        \ifx\tok@AmS\rq\srdr@AmS'\else \drsr@AmS'\fi \fi
        \fi}



\def\dash{\unskip\penalty0---}
\def\slash{\unskip/\penalty\exhyphenpenalty}

\def\{{\relax@AmS\ifmmode\delimiter"4266308 \else
    $\delimiter"4266308 $\fi}                            % NOTE 16 (No NOTE 15)

\def\}{\relax@AmS\ifmmode\delimiter"5267309 \else$\delimiter"5267309 $\fi}


\def\AmSTeX{$\cal A$\kern-.1667em\lower.5ex\hbox{$\cal M$}\kern-.125em
     $\cal S$-\TeX}

\def\smallvspace{\vskip\smallskipamount}                            % NOTE 17
\def\medvspace{\vskip\medskipamount}
\def\bigvspace{\vskip\bigskipamount}


\def\linebreak{\unskip\penalty-10000 }                                % NOTE 18
\def\pagebreak{\vadjust{\penalty-10000 }}
\def\newpage{\par\vfill\eject}

\def\newline{\ifvmode \err@AmS{There's no line here to break}\else
     \hfil\penalty-10000\fi}

\def\topspace#1{\insert\topins{\penalty100 \splittopskip=0pt
     \vbox to #1{}}}
\def\midspace#1{\setbox0=\vbox to #1{}\advance\dimen0 by \pagetotal
  \ifdim\dimen0>\pagegoal\topspace{#1}\else\vadjust{\box0}\fi}

\long\def\comment{\begingroup
 \catcode`\{=12 \catcode`\}=12 \catcode`\#= 12 \catcode`\^^M=12
   \catcode`\%=12 \catcode`^^A=14
    \comment@AmS}
\begingroup\catcode`^^A=14
\catcode`\^^M=12  ^^A
\long\gdef\comment@AmS#1^^M#2{\comp@AmS\endcomment{#2}\if T\cresult@AmS^^A
\def\comment@@AmS{\endgroup}\else^^A
 \long\def\comment@@AmS{\comment@AmS#2}\fi\comment@@AmS}\endgroup     % NOTE 19


% STYLE, SPACING AND ALTERNATE NAMES

\let\dsize=\displaystyle
\let\tsize=\textstyle
\let\ssize=\scriptstyle
\let\sssize=\scriptscriptstyle

\def\text#1{\hbox{\rm#1}}

\def\quad{\relax@AmS\ifmmode
    \hbox{\hskip1em}\else\hskip1em\relax@AmS\fi}                      % NOTE 20
\def\qquad{\quad\quad}
\def\,{\relax@AmS\ifmmode\mskip\thinmuskip\else$\mskip\thinmuskip$\fi}
\def\;{\relax@AmS
  \ifmmode\mskip\thickmuskip\else$\mskip\thickmuskip$\fi}


\def\stack#1#2{{#1\atop #2}}

\def\frac#1#2{{#1\over#2}}
\def\Frac#1#2{{\displaystyle{#1\over#2}}}

\def\thickfrac#1#2{{#1\above1pt #2}}

\def\binom#1#2{{#1\choose #2}}
\def\Binom#1#2{{\displaystyle{#1\choose #2}}}


\mathchardef\:="603A                                                  % NOTE 21


% BIG DELIMITERS

\def\big@AmS#1{{\hbox{$\left#1\vbox to\big@@AmS{}\right.\offspace@AmS$}}}
\def\Big@AmS#1{{\hbox{$\left#1\vbox to\Big@@AmS{}\right.\offspace@AmS$}}}
\def\bigg@AmS#1{{\hbox{$\left#1\vbox to\bigg@@AmS{}\right.\offspace@AmS$}}}
\def\Bigg@AmS#1{{\hbox{$\left#1\vbox to\Bigg@@AmS{}\right.\offspace@AmS$}}}
\def\offspace@AmS{\nulldelimiterspace0pt \mathsurround0pt }

\def\big@@AmS{8.5pt}                                % NOTE 24 (no NOTES 22, 23)
\def\Big@@AmS{11.5pt}
\def\bigg@@AmS{14.5pt}
\def\Bigg@@AmS{17.5pt}

\def\bigl{\mathopen\big@AmS}
\def\bigm{\mathrel\big@AmS}
\def\bigr{\mathclose\big@AmS}
\def\Bigl{\mathopen\Big@AmS}
\def\Bigm{\mathrel\Big@AmS}
\def\Bigr{\mathclose\Big@AmS}
\def\biggl{\mathopen\bigg@AmS}
\def\biggm{\mathrel\bigg@AMS}
\def\biggr{\mathclose\bigg@AmS}
\def\Biggl{\mathopen\Bigg@AmS}
\def\Biggm{\mathrel\Bigg@AmS}
\def\Biggr{\mathclose\Bigg@AmS}


%  MAKING ' WORK FOR PRIMES

{\catcode`'=13 \gdef'{^\bgroup\prime\prime@AmS}}
\def\prime@AmS{\futurelet\tok@AmS\prime@@AmS}
\def\prime@@@AmS#1{\futurelet\tok@AmS\prime@@AmS}
\def\prime@@AmS{\ifx\tok@AmS'\def\next@AmS{\prime\prime@@@AmS}\else
   \def\next@AmS{\egroup}\fi\next@AmS}


%  SMASHES                                                            % NOTE 25

\def\topsmash{\relax@AmS\ifmmode\def\topsmash@AmS
   {\mathpalette\mathtopsmash@AmS}\else
    \let\topsmash@AmS=\maketopsmash@AmS\fi\topsmash@AmS}

\def\maketopsmash@AmS#1{\setbox0=\hbox{#1}\topsmash@@AmS}

\def\mathtopsmash@AmS#1#2{\setbox0=\hbox{$#1{#2}$}\topsmash@@AmS}

\def\topsmash@@AmS{\vbox to 0pt{\kern-\ht0\box0}}


\def\botsmash{\relax@AmS\ifmmode\def\botsmash@AmS
   {\mathpalette\mathbotsmash@AmS}\else
     \let\botsmash@AmS=\makebotsmash@AmS\fi\botsmash@AmS}

\def\makebotsmash@AmS#1{\setbox0=\hbox{#1}\botsmash@@AmS}

\def\mathbotsmash@AmS#1#2{\setbox0=\hbox{$#1{#2}$}\botsmash@@AmS}

\def\botsmash@@AmS{\vbox to \ht0{\box0\vss}}



%  LARGE OPERATORS

\def\LimitsOnSums{\let\slimits@AmS=\displaylimits}                    % NOTE 26
\def\NoLimitsOnSums{\let\slimits@AmS=\nolimits}

\LimitsOnSums

\mathchardef\coprod@AmS"1360       \def\coprod{\coprod@AmS\slimits@AmS}
\mathchardef\bigvee@AmS"1357       \def\bigvee{\bigvee@AmS\slimits@AmS}
\mathchardef\bigwedge@AmS"1356     \def\bigwedge{\bigwedge@AmS\slimits@AmS}
\mathchardef\biguplus@AmS"1355     \def\biguplus{\biguplus@AmS\slimits@AmS}
\mathchardef\bigcap@AmS"1354       \def\bigcap{\bigcap@AmS\slimits@AmS}
\mathchardef\bigcup@AmS"1353       \def\bigcup{\bigcup@AmS\slimits@AmS}
\mathchardef\prod@AmS"1351         \def\prod{\prod@AmS\slimits@AmS}
\mathchardef\sum@AmS"1350          \def\sum{\sum@AmS\slimits@AmS}
\mathchardef\bigotimes@AmS"134E    \def\bigotimes{\bigotimes@AmS\slimits@AmS}
\mathchardef\bigoplus@AmS"134C     \def\bigoplus{\bigoplus@AmS\slimits@AmS}
\mathchardef\bigodot@AmS"134A      \def\bigodot{\bigodot@AmS\slimits@AmS}
\mathchardef\bigsqcup@AmS"1346     \def\bigsqcup{\bigsqcup@AmS\slimits@AmS}


\def\LimitsOnInts{\let\ilimits@AmS=\displaylimits}
\def\NoLimitsOnInts{\let\ilimits@AmS=\nolimits}

\NoLimitsOnInts

\mathchardef\int@AmS"1352
\def\int{\gdef\intflag@AmS{T}\int@AmS\ilimits@AmS}                    % NOTE 27

\mathchardef\oint@AmS"1348 \def\oint{\gdef\intflag@AmS{T}\oint@AmS\ilimits@AmS}

\def\inttest@AmS#1{\def\intflag@AmS{F}\setbox0=\hbox{$#1$}}


\def\intic@AmS{\mathchoice{\hbox{\hskip5pt}}{\hbox
          {\hskip4pt}}{\hbox{\hskip4pt}}{\hbox{\hskip4pt}}}           % NOTE 28
\def\negintic@AmS{\mathchoice
  {\hbox{\hskip-5pt}}{\hbox{\hskip-4pt}}{\hbox{\hskip-4pt}}{\hbox{\hskip-4pt}}}
\def\intkern@AmS{\mathchoice{\!\!\!}{\!\!}{\!\!}{\!\!}}
\def\intdots@AmS{\mathchoice{\cdots}{{\cdotp}\mkern 1.5mu
    {\cdotp}\mkern 1.5mu{\cdotp}}{{\cdotp}\mkern 1mu{\cdotp}\mkern 1mu
      {\cdotp}}{{\cdotp}\mkern 1mu{\cdotp}\mkern 1mu{\cdotp}}}

\newcount\intno@AmS                                                   % NOTE 29

\def\intii{\gdef\intflag@AmS{T}\intno@AmS=2\futurelet                 % NOTE 30
              \tok@AmS\ints@AmS}
\def\intiii{\gdef\intflag@AmS{T}\intno@AmS=3\futurelet\tok@AmS\ints@AmS}
\def\intiv{\gdef\intflag@AmS{T}\intno@AmS=4\futurelet\tok@AmS\ints@AmS}
\def\intdotsint{\gdef\intflag@AmS{T}\intno@AmS=0\futurelet
    \tok@AmS\ints@AmS}

\def\ints@AmS{\findlimits@AmS\ints@@AmS}

\def\findlimits@AmS{\def\ignoretoken@AmS{T}\ifx\tok@AmS\limits
   \def\limits@AmS{T}\else\ifx\tok@AmS\nolimits\def\limits@AmS{F}\else
     \def\ignoretoken@AmS{F}\ifx\ilimits@AmS\nolimits\def\limits@AmS{F}\else
       \def\limits@AmS{T}\fi\fi\fi}

\def\multintlimits@AmS{\int@AmS\ifnum \intno@AmS=0\intdots@AmS
  \else \intkern@AmS\fi
    \ifnum\intno@AmS>2\int@AmS\intkern@AmS\fi
     \ifnum\intno@AmS>3 \int@AmS\intkern@AmS\fi \int@AmS}

\def\multint@AmS{\int\ifnum \intno@AmS=0\intdots@AmS\else\intkern@AmS\fi
   \ifnum\intno@AmS>2\int\intkern@AmS\fi
    \ifnum\intno@AmS>3 \int\intkern@AmS\fi \int}

\def\ints@@AmS{\if F\ignoretoken@AmS\def\ints@@@AmS{\if
    T\limits@AmS\negintic@AmS
 \mathop{\intic@AmS\multintlimits@AmS}\limits\else
    \multint@AmS\nolimits\fi}\else\def\ints@@@AmS{\if T\limits@AmS
   \negintic@AmS\mathop{\intic@AmS\multintlimits@AmS}\limits\else
    \multint@AmS\nolimits\fi\eat@AmS}\fi\ints@@@AmS}



\def\LimitsOnNames{\let\nlimits@AmS=\displaylimits}
\def\NoLimitsOnNames{\let\nlimits@AmS=\nolimits}

\LimitsOnNames

\def\operatorname#1{\mathop{\mathcode`'="7027 \mathcode`-="70
       \rm #1}\nolimits}                                              % NOTE 31

\def\operatornamewithlimits#1{\mathop{\mathcode`'="7027 \mathcode`-="702D
   \rm #1}\nlimits@AmS}

\def\operator#1{\mathop{#1}\nolimits}
\def\operatorwithlimits#1{\mathop{#1}\displaylimits}

\def\limover{\mathop{\overline{\rm lim}}}
\def\limunder{\mathop{\underline{\vrule height 0pt depth .2ex width 0pt
       \rm lim}}}
\def\liminj{\setbox0=\hbox{\rm lim}\mathop{\rm lim}
		\limits_{\topsmash{\hbox to \wd0{\leftarrowfill}}}}
\def\limproj{\setbox0=\hbox{\rm lim}\mathop{\rm lim}
		\limits_{\topsmash{\hbox to \wd0{\rightarrowfill}}}}


% SUBSIDIARY CONSIDERATIONS FOR LARGE OPERATORS -- BUFFER AND SHAVE

\newdimen\buffer@AmS
\buffer@AmS=\fontdimen13\tenex                                        % NOTE 32
\newdimen\buffer
\buffer=\buffer@AmS

\def\changebuffer#1{\fontdimen13 \tenex=#1 \buffer=\fontdimen13 \tenex}
\def\resetbuffer{\fontdimen13 \tenex=\buffer@AmS \buffer=\buffer@AmS}

\def\shave#1{\mathop{\hbox{$\fontdimen13\tenex=0pt                    % NOTE 33
     \displaystyle{#1}$}}\fontdimen13\tenex=1\buffer}


\def\topshave#1{\topsmash{#1}\vphantom{\shave{#1}}}

\def\botshave#1{\botsmash{#1}\vphantom{\shave{#1}}}



% ALIGNED UNITS

\def\Let@AmS{\relax@AmS\iffalse{\fi\let\\=\cr\iffalse}\fi}            % NOTE 34

\def\align{\def\vspace##1{\noalign{\vskip ##1}}                       % NOTE 35
  \,\vcenter\bgroup\Let@AmS\tabskip=0pt\openup3pt\mathsurround=0pt  % NOTE 35.1
  \halign\bgroup\strut
  \hfil$\displaystyle{##}$&$\displaystyle{{}##}$\hfil\cr}        % NOTES 36, 37

\def\endalign{\strut\crcr\egroup\egroup}


\def\bunch{\def\vspace##1{\noalign{\vskip ##1}}
  \,\vcenter\bgroup\Let@AmS\tabskip=0pt\openup3pt\mathsurround=0pt
     \halign\bgroup\strut\hfil$\displaystyle{##}$\hfil\cr}

\def\endbunch{\strut\crcr\egroup\egroup}

\def\matrix{\catcode`\^^I=4 \futurelet\tok@AmS\matrix@AmS}            % NOTE 38

\def\matrix@AmS{\relax@AmS\iffalse{\fi \ifnum`}=0\fi\ifx\tok@AmS\format
   \def\next@AmS{\expandafter\matrix@@AmS\eat@AmS}\else
   \def\next@AmS{\matrix@@@AmS}\fi\next@AmS}

\def\matrix@@@AmS{
 \ifnum`{=0\fi\iffalse}\fi\,\vcenter\bgroup\Let@AmS\tabskip=0pt
    \normalbaselines\halign\bgroup $\strut\hfil##\hfil$&&\quad$\strut
  \hfil##\hfil$\cr\strut\cr\noalign{\kern-\baselineskip}}             % NOTE 39

\def\matrix@@AmS#1\\{
   \def\premable@AmS{#1}\toks@{##}
 \def\c{$\copy\strutbox\hfil\the\toks@\hfil$}\def\r
   {$\copy\strutbox\hfil\the\toks@$}%
   \def\l{$\copy\strutbox\the\toks@\hfil$}%
\setbox0=
\hbox{\xdef\Preamble@AmS{\premable@AmS}}
 \def\vspace##1{\noalign{\vskip ##1}}\ifnum`{=0\fi\iffalse}\fi
\,\vcenter\bgroup\Let@AmS
  \tabskip=0pt\normalbaselines\halign\bgroup\span\Preamble@AmS\cr
    \mathstrut\cr\noalign{\kern-\baselineskip}}
                                                                      % NOTE 40

\def\endmatrix{\crcr\mathstrut\cr\noalign{\kern-\baselineskip
   }\egroup\egroup\,\catcode`\^^I=10 }



\def\matrixp{\left(\matrix}
\def\endmatrixp{\endmatrix\right)}

\def\matrixb{\left[\matrix}
\def\endmatrixb{\endmatrix\right]}

\def\matrixv{\left|\matrix}
\def\endmatrixv{\endmatrix\right|}

\def\matrixvv{\left\|\matrix}
\def\endmatrixvv{\endmatrix\right\|}


\def\spacedots#1for#2{\multispan#2\leaders\hbox{$\mkern#1mu.\mkern
    #1mu$}\hfill}
\def\dotsfor#1{\spacedots 1.5 for #1}                                 % NOTE 41


\def\enabletabs{\catcode`\^^I=4 \enabletabs@AmS}
\def\enabletabs@AmS#1\disabletabs{#1\catcode`\^^I=10 }                % NOTE 42

\def\Enabletabs{\catcode`\^^I=4 }
\def\Disabletabs{\catcode`\^^I=10 }


\def\smallmatrix{\futurelet\tok@AmS\smallmatrix@AmS}                  % NOTE 43

\def\smallmatrix@AmS{\relax@AmS\iffalse{\fi \ifnum`}=0\fi\ifx\tok@AmS\format
   \def\next@AmS{\expandafter\smallmatrix@@AmS\eat@AmS}\else
   \def\next@AmS{\smallmatrix@@@AmS}\fi\next@AmS}

\def\smallmatrix@@@AmS{
 \ifnum`{=0\fi\iffalse}\fi\,\vcenter\bgroup\Let@AmS\tabskip=0pt
    \baselineskip8pt\lineskip1pt\lineskiplimit1pt
  \halign\bgroup $\strut\hfil##\hfil$&&\;$\strut
  \hfil##\hfil$\cr\strut\cr\noalign{\kern-\baselineskip}}

\def\smallmatrix@@AmS#1\\{
   \def\premable@AmS{#1}\toks@{##}
 \def\c{$\copy\strutbox\hfil\the\toks@\hfil$}\def\r
   {$\copy\strutbox\hfil\the\toks@$}%
   \def\l{$\copy\strutbox\the\toks@\hfil$}%
\hbox{\xdef\Preamble@AmS{\premable@AmS}}
 \def\vspace##1{\noalign{\vskip ##1}}\ifnum`{=0\fi\iffalse}\fi
\,\vcenter\bgroup\Let@AmS
     \tabskip=0pt\baselineskip8pt\lineskip1pt\lineskiplimit1pt
\halign\bgroup\span\Preamble@AmS\cr
    \mathstrut\cr\noalign{\kern-\baselineskip}}

\def\endsmallmatrix{\crcr\mathstrut\cr\noalign{\kern-\baselineskip}
   \egroup\egroup\,}


\def\cases{\left\{ \,\vcenter\bgroup\Let@AmS\normalbaselines\tabskip=0pt
   \halign\bgroup$##\hfil$&\qquad$##\hfil$\cr}                        % NOTE 44

\def\endcases{\crcr\egroup\egroup\right.}


% TAGGING

\def\TagsOnLeft{\def\tagposition@AmS{L}}
\def\TagsOnRight{\def\tagposition@AmS{R}}
\def\TagsAsMath{\def\tagstyle@AmS{M}}
\def\TagsAsText{\def\tagstyle@AmS{T}}

\TagsOnLeft
\TagsAsText

\def\tag#1$${\if L\tagposition@AmS
    \leqno\else\eqno\fi\def\atag@AmS{T}\maketag@AmS#1\tagend@AmS$$}   % NOTE 45

\def\maketag@AmS{\futurelet\tok@AmS\maketag@@AmS}                     % NOTE 46
\def\maketag@@AmS{\ifx\tok@AmS[\def\next@AmS{\maketag@@@AmS}\else
      \def\next@AmS{\maketag@@@@AmS}\fi\next@AmS}
\def\maketag@@@AmS[#1]#2\tagend@AmS{\if F\atag@AmS\else             % NOTE 46.1
   \if M\tagstyle@AmS\hbox{$#1$}\else\hbox{#1}\fi\fi
       \gdef\atag@AmS{F}}
\def\maketag@@@@AmS#1\tagend@AmS{\if F\atag@AmS \else
        \if T\autotag@AmS \setbox0=\hbox
    {\if M\tagstyle@AmS\tagform@AmS{$#1$}\else\tagform@AmS{#1}\fi}
                        \ifdim\wd0=0pt \tagform@AmS{*}\else
            \if M\tagstyle@AmS\tagform@AmS{$#1$}\else\tagform@AmS{#1}\fi
                     \fi\else
               \if M\tagstyle@AmS\tagform@AmS{$#1$}\else\tagform@AmS{#1}\fi
                     \fi
                  \fi\gdef\atag@AmS{F}}

\def\tagform@AmS#1{\hbox{\rm(#1\unskip)}}

\def\AutoTag{\def\autotag@AmS{T}}
\def\NoAutoTag{\def\autotag@AmS{F}}

\NoAutoTag





\def\inaligntag@AmS{F} \def\inbunchtag@AmS{F}                         % NOTE 47

\def\CenteredTagsOnBrokens{\def\centerbroken@AmS{T}}                  % NOTE 48
\def\TopOrBottomTagsOnBrokens{\def\centerbroken@AmS{F}}
\TopOrBottomTagsOnBrokens

\def\broken{\global\setbox0=\vbox\bgroup\Let@AmS\tabskip=0pt
 \if T\inaligntag@AmS\else
   \if T\inbunchtag@AmS\else\openup3pt\fi\fi\mathsurround=0pt
     \halign\bgroup\strut\hfil$\displaystyle{##}$&$\displaystyle{{}##}$\hfill
      \cr}
                                                                      % NOTE 49
\def\endbroken{\strut\crcr\egroup\egroup
      \global\setbox7=\vbox{\unvbox0\setbox1=\lastbox
      \hbox{\unhbox1\unskip\setbox2=\lastbox
       \unskip\setbox3=\lastbox
         \global\setbox4=\copy3
          \box3\box2}}%                                               % NOTE 50
  \if L\tagposition@AmS
     \if T\inaligntag@AmS
           \if T\centerbroken@AmS\gdef\broken@AmS
                {&\vcenter{\vbox{\moveleft\wd4\box7}}}%               % NOTE 51
           \else
            \gdef\broken@AmS{&\vbox{\moveleft\wd4\vtop{\unvbox7}}}%   % NOTE 52
           \fi
     \else                                                            % NOTE 53
           \if T\centerbroken@AmS\gdef\broken@AmS
                {\vcenter{\box7}}%
           \else
              \gdef\broken@AmS{\vtop{\unvbox7}}%
           \fi
     \fi
  \else                                                  % NOTE 55 (no note 54)
      \if T\inaligntag@AmS
           \if T\centerbroken@AmS
              \gdef\broken@AmS{&\vcenter{\vbox{\moveleft\wd4\box7}}}%
          \else
             \gdef\broken@AmS{&\vbox{\moveleft\wd4\box7}}%
          \fi
      \else
          \if T\centerbroken@AmS
            \gdef\broken@AmS{\vcenter{\box7}}%
          \else
             \gdef\broken@AmS{\box7}%
          \fi
      \fi
  \fi\broken@AmS}

\def\cbroken{\xdef\centerbroken@@AmS{\centerbroken@AmS}%
                       \def\centerbroken@AmS{T}\broken}               % NOTE 56
\def\endcbroken{\endbroken\def\centerbroken@AmS{\centerbroken@@AmS}}


\def\multline#1${\in@AmS\tag{#1}\if T\cresult@AmS
 \def\multline@AmS{\def\atag@AmS{T}\getmltag@AmS#1$}\else
   \def\multline@AmS{\def\atag@AmS{F}\setbox9=\hbox{}\multline@@AmS
    \multline@@@AmS#1$}\fi\multline@AmS}                              % NOTE 57

\def\getmltag@AmS#1\tag#2${\setbox9=\hbox{\maketag@AmS#2\tagend@AmS}%
           \multline@@AmS\multline@@@AmS#1$}

\def\multline@@AmS{\if L\tagposition@AmS
     \def\lwidth@AmS{\hskip\wd9}\def\rwidth@AmS{\hskip0pt}\else
      \def\lwidth@AmS{\hskip0pt}\def\rwidth@AmS{\hskip\wd9}\fi}      % NOTE 58

\def\multline@@@AmS{\def\vspace##1{\noalign{\vskip ##1}}%
 \def\shoveright##1{##1\hfilneg\rwidth@AmS\quad}                      % NOTE 59
  \def\shoveleft##1{\setbox                                           % NOTE 60
      0=\hbox{$\displaystyle{}##1$}%
     \setbox1=\hbox{$\displaystyle##1$}%
     \ifdim\wd0=\wd1
    \hfilneg\lwidth@AmS\quad##1\else
      \setbox2=\hbox{\hskip\wd0\hskip-\wd1}%
       \hfilneg\lwidth@AmS\quad\hskip-.5\wd2 ##1\fi}
     \vbox\bgroup\Let@AmS\openup3pt\halign\bgroup\hbox to \the\displaywidth
      {$\displaystyle\hfil{}##\hfil$}\cr\hfilneg\quad
      \if L\tagposition@AmS\hskip-1em\copy9\quad\else\fi}             % NOTE 61


\def\endmultline{\if R\tagposition@AmS\quad\box9                 % NOTES 62, 63
   \hskip-1em\else\fi\quad\hfilneg\crcr\egroup\egroup}



\def\aligntag#1$${\def\inaligntag@AmS{T}\openup3pt\mathsurround=0pt   % NOTE 64
 \Let@AmS
   \def\tag{\gdef\atag@AmS{T}&}                                       % NOTE 65
   \def\vspace##1{\noalign{\vskip##1}}                                % NOTE 66
    \def\xtext##1{\noalign{\hbox{##1}}}                               % NOTE 67
   \def\break{\noalign{\penalty-10000 }}                              % NOTE 68
   \def\nobreak{\noalign{\penalty 10000 }}
   \def\allowbreak{\noalign{\penalty 0 }}
   \def\goodbreak{\noalign{\penalty -500 }}
    \gdef\atag@AmS{F}%
\if L\tagposition@AmS\laligntag@AmS#1$$\else
   \raligntag@AmS#1$$\fi}

\def\raligntag@AmS#1$${\tabskip\centering
   \halign to \the\displaywidth
{\hfil$\displaystyle{##}$\tabskip 0pt
    &$\displaystyle{{}##}$\hfil\tabskip\centering
   &\llap{\maketag@AmS##\tagend@AmS}\tabskip 0pt\cr\noalign{\vskip-
     \lineskiplimit}#1\crcr}$$}

\def\laligntag@AmS#1$${\tabskip\centering                             % NOTE 69
   \halign to \the\displaywidth
{\hfil$\displaystyle{##}$\tabskip0pt
   &$\displaystyle{{}##}$\hfil\tabskip\centering
    &\kern-\displaywidth\rlap{\maketag@AmS##\tagend@AmS}\tabskip
    \the\displaywidth\cr\noalign{\vskip-\lineskiplimit}#1\crcr}$$}

\def\endaligntag{}

\def\bunchtag#1$${\def\inbunchtag@AmS{T}\openup3pt\mathsurround=0pt   % NOTE 70
    \Let@AmS
   \def\tag{\gdef\atag@AmS{T}&}
   \def\vspace##1{\noalign{\vskip##1}}
   \def\xtext##1{\noalign{\hbox{##1}}}
   \def\break{\noalign{\penalty-10000 }}
   \def\nobreak{\noalign{\penalty 10000 }}
   \def\allowbreak{\noalign{\penalty 0 }}
    \def\goodbreak{\noalign{\penalty -500 }}
  \if L\tagposition@AmS\lbunchtag@AmS#1$$\else
    \rbunchtag@AmS#1$$\fi}

\def\rbunchtag@AmS#1$${\tabskip\centering
    \halign to \displaywidth {$\hfil\displaystyle{##}\hfil$&
      \llap{\maketag@AmS##\tagend@AmS}\tabskip 0pt\cr\noalign{\vskip-
       \lineskiplimit}#1\crcr}$$}

\def\lbunchtag@AmS#1$${\tabskip\centering
   \halign to \displaywidth
{$\hfil\displaystyle{##}\hfil$&\kern-
    \displaywidth\rlap{\maketag@AmS##\tagend@AmS}\tabskip\the\displaywidth\cr
    \noalign{\vskip-\lineskiplimit}#1\crcr}$$}

\def\endbunchtag{}


%  MISCELLANEOUS

\def\hyphen{\mathchar"702D}                                           % NOTE 71
\def\endash{\mathchar"707B}
\def\emdash{\mathchar"707C}
\def\rightquote{\mathchar"7027}
\def\rightquoteii{\mathchar"7022}
\def\leftquote{\mathchar"7060}
\def\leftquoteii{\mathchar"705C}

\def\mod#1{\allowbreak\mkern18mu{\rm mod}\,\,#1}

% CONTINUED FRACTIONS

\def\numeratorleft#1{#1\hskip 0pt plus 1filll\relax@AmS}
\def\numeratorright#1{\hskip 0pt plus 1filll\relax@AmS#1}
\def\numeratorcenter#1{\hskip 0pt plus 1filll\relax@AmS
      #1\hskip 0pt plus 1filll\relax@AmS}

\def\cfrac@AmS#1,{\def\numerator@AmS{#1}\cfrac@@AmS*}                 % NOTE 72

\def\cfrac@@AmS#1;#2#3\cfend@AmS{\comp@AmS\cfmark@AmS{#2}\if T\cresult@AmS
 \gdef\cfrac@@@AmS
  {\expandafter\eat@AmS\numerator@AmS\strut\over\eat@AmS#1}\else
  \comp@AmS;{#2}\if T\cresult@AmS\gdef\cfrac@@@AmS
  {\expandafter\eat@AmS\numerator@AmS\strut\over\eat@AmS#1}\else
\gdef\cfrac@@@AmS{\if R\cftype@AmS\hfill\else\fi
    \expandafter\eat@AmS\numerator@AmS\strut
    \if L\cftype@AmS\hfill\else\fi\over
       \eat@AmS#1\displaystyle {\cfrac@AmS*#2#3\cfend@AmS}}
     \fi\fi\cfrac@@@AmS}

\def\cfrac#1{\def\cftype@AmS{C}\cfrac@AmS*#1;\cfmark@AmS\cfend@AmS}

\def\cfracl#1{\def\cftype@AmS{L}\cfrac@AmS*#1;\cfmark@AmS\cfend@AmS}

\def\cfracr#1{\def\cftype@AmS{R}\cfrac@AmS*#1;\cfmark@AmS\cfend@AmS}

\def\adorn#1#2#3{\mathsurround=0pt\setbox0=\hbox{$\displaystyle{#2}#3$}%
   \setbox1=\hbox{$\displaystyle\vphantom{#2}#1{#2}$}%
    \setbox2=\hbox{\hskip\wd0\hskip-\wd1}%
    \hskip-\wd2\mathop{\hskip\wd2\vphantom{#2}#1{#2}#3}}

%  ARROWS                                                             % NOTE 73

\def\overrightarrow{\mathpalette\overrightarrow@AmS}

\def\overrightarrow@AmS#1#2{\vbox{\halign{$##$\cr
    #1{-}\mkern-6mu\cleaders\hbox{$#1\mkern-2mu{-}\mkern-2mu$}\hfill
     \mkern-6mu{\to}\cr
     \noalign{\kern -1pt\nointerlineskip}
     \hfil#1#2\hfil\cr}}}

\let\overarrow=\overrightarrow

\def\overleftarrow{\mathpalette\overleftarrow@Ams}

\def\overleftarrow@Ams#1#2{\vbox{\halign{$##$\cr
     #1{\leftarrow}\mkern-6mu\cleaders\hbox{$#1\mkern-2mu{-}\mkern-2mu$}\hfill
      \mkern-6mu{-}\cr
     \noalign{\kern -1pt\nointerlineskip}
     \hfil#1#2\hfil\cr}}}

\def\overleftrightarrow{\mathpalette\overleftrightarrow@AmS}

\def\overleftrightarrow@AmS#1#2{\vbox{\halign{$##$\cr
     #1{\leftarrow}\mkern-6mu\cleaders\hbox{$#1\mkern-2mu{-}\mkern-2mu$}\hfill
       \mkern-6mu{\to}\cr
    \noalign{\kern -1pt\nointerlineskip}
      \hfil#1#2\hfil\cr}}}

\def\underrightarrow{\mathpalette\underrightarrow@AmS}

\def\underrightarrow@AmS#1#2{\vtop{\halign{$##$\cr
    \hfil#1#2\hfil\cr
     \noalign{\kern -1pt\nointerlineskip}
    #1{-}\mkern-6mu\cleaders\hbox{$#1\mkern-2mu{-}\mkern-2mu$}\hfill
     \mkern-6mu{\to}\cr}}}

\let\underarrow=\underrightarrow

\def\underleftarrow{\mathpalette\underleftarrow@AmS}

\def\underleftarrow@AmS#1#2{\vtop{\halign{$##$\cr
     \hfil#1#2\hfil\cr
     \noalign{\kern -1pt\nointerlineskip}
     #1{\leftarrow}\mkern-6mu\cleaders\hbox{$#1\mkern-2mu{-}\mkern-2mu$}\hfill
      \mkern-6mu{-}\cr}}}

\def\underleftrightarrow{\mathpalette\underleftrightarrow@AmS}

\def\underleftrightarrow@AmS#1#2{\vtop{\halign{$##$\cr
      \hfil#1#2\hfil\cr
    \noalign{\kern -1pt\nointerlineskip}
     #1{\leftarrow}\mkern-6mu\cleaders\hbox{$#1\mkern-2mu{-}\mkern-2mu$}\hfill
       \mkern-6mu{\to}\cr}}}

% DOTS

\def\dotsc{\mathinner{\ldotp\ldotp\ldotp}}
\def\dotsi{\mathinner{\cdotp\cdotp\cdotp}}
\def\dotsj{\mathinner{\ldotp\ldotp\ldotp}}
\def\dotsb{\mathinner{\cdotp\cdotp\cdotp}}

\def\binary@AmS#1{{\thinmuskip 0mu \medmuskip 1mu \thickmuskip 1mu    % NOTE 74
      \setbox0=\hbox{$#1{}{}{}{}{}{}{}{}{}$}\setbox1=\hbox
       {${}#1{}{}{}{}{}{}{}{}{}$}\ifdim\wd1>\wd0\gdef\binary@@AmS{T}\else
       \gdef\binary@@AmS{F}\fi}}

\def\dots{\relax@AmS\ifmmode\def\dots@AmS{\mdots@AmS}\else
    \def\dots@AmS{\tdots@AmS}\fi\dots@AmS}

\def\mdots@AmS{\futurelet\tok@AmS\mdots@@AmS}

\def\mdots@@AmS{\def\thedots@AmS{\dotsj}%
  \ifx\tok@AmS\bgroup\else
  \ifx\tok@AmS\egroup\else
  \ifx\tok@AmS$\else
  \iffalse{\fi  \ifx\tok@AmS\\ \iffalse}\fi\else                      % NOTE 75
  \iffalse{\fi \ifx\tok@AmS&  \iffalse}\fi\else
  \ifx\tok@AmS\left\else
  \ifx\tok@AmS\right\else
  \ifx\tok@AmS,\def\thedots@AmS{\dotsc}\else
  \inttest@AmS\tok@AmS\if T\intflag@AmS\def\thedots@AmS{\dotsi}\else
  \binary@AmS\tok@AmS\if T\binary@@AmS\def\thedots@AmS{\dotsb}\else
   \def\thedots@AmS{\dotsj}\fi\fi\fi\fi\fi\fi\fi\fi\fi\fi\thedots@AmS}

\def\tdots@AmS{\unskip\ \tdots@@AmS}

\def\tdots@@AmS{\futurelet\tok@AmS\tdots@@@AmS}

\def\tdots@@@AmS{$\ldots\,
   \ifx\tok@AmS,$\else
   \ifx\tok@AmS.\,$\else
   \ifx\tok@AmS;\,$\else
   \ifx\tok@AmS:\,$\else
   \ifx\tok@AmS?\,$\else
   \ifx\tok@AmS!\,$\else
   $\ \fi\fi\fi\fi\fi\fi}


% SET NOTATION

\def\lset{\{\,}
\def\rset{\,\}}


\def\leftset#1\mid#2\rightset{\hbox{$\displaystyle
\left\{\,#1\vphantom{#1#2}\;\right|\;\left.
    #2\vphantom{#1#2}\,\right\}\offspace@AmS$}}


% ACCENT SYMBOLS

\def\dotii#1{{\mathop{#1}\limits^{\vbox to -1.4pt{\kern-2pt
   \hbox{\tenrm..}\vss}}}}
\def\dotiii#1{{\mathop{#1}\limits^{\vbox to -1.4pt{\kern-2pt
   \hbox{\tenrm...}\vss}}}}
\def\dotiv#1{{\mathop{#1}\limits^{\vbox to -1.4pt{\kern-2pt
   \hbox{\tenrm....}\vss}}}}

\def\vecsymbol{\rightarrow}
\def\barsymbol{-}
\def\tildesymbol{\mathchar"0218 }
\def\hatsymbol{{\mathchoice{\null}{\null}{\,\,\hbox{\lower 10pt\hbox
    {$\widehat{\null}$}}}{\,\hbox{\lower 20pt\hbox
       {$\hat{\null}$}}}}}
\def\dotsymbol{{\nonscript\,.}}
\def\dotiisymbol{{\nonscript\,\hbox{\tenrm..}}}
\def\dotiiisymbol{{\nonscript\,\hbox{\tenrm...}}}
\def\dotivsymbol{{\nonscript\,\hbox{\tenrm....}}}
\def\dotsymbol{{\nonscript\,\hbox{\tenrm.}}}



% OVERSET AND OVERBRACE

\def\overset#1\to#2{{\mathop{#2}^{#1}}}

\def\underset#1\to#2{{\mathop{#2}_{#1}}}

\def\oversetbrace#1\to#2{{\overbrace{#2}^{#1}}}
\def\undersetbrace#1\to#2{{\underbrace{#2}_{#1}}}


% ROOTS

\def\uproot#1{\gdef\theuproot{#1 pt}}
\def\theuproot{0 pt}

\def\therightroot{0mu}
\def\rightroot#1{\gdef\therightroot{-#1mu}}


\def\r@@t#1#2{\setbox\z@\hbox{$\m@th#1\sqrt{#2}$}%
  \dimen@\ht\z@ \advance\dimen@-\dp\z@ \advance\dimen@\theuproot
  \mskip5mu\raise.6\dimen@\copy\rootbox \mskip-10mu \mskip\therightroot
    \box\z@\gdef\theuproot{0 pt}\gdef\therightroot{0mu}}              % NOTE 76


%  BOXED


\def\boxed#1{\setbox0=\hbox{$\displaystyle{#1}$}\hbox{\lower.4pt\hbox{\lower
   3pt\hbox{\lower 1\dp0\hbox{\vbox{\hrule height .4pt \hbox{\vrule width
   .4pt \hskip 3pt\vbox{\vskip 3pt\box0\vskip3pt}\hskip 3pt \vrule width
      .4pt}\hrule height .4pt}}}}}}

%  FORMATTING MACROS COMMON TO ALL STYLES

\def\documentstyle#1{\input #1.sty}

\newif\ifretry@AmS
\def\y@AmS{y } \def\y@@AmS{Y } \def\n@AmS{n } \def\n@@AmS{N }
\def\ask@AmS{\message
  {Do you want output? (y or n, follow answer by return) }\loop
   \read-1 to\answer@AmS
  \ifx\answer@AmS\y@AmS\retry@AmSfalse\outputon
   \else\ifx\answer@AmS\y@@AmS\retry@AmSfalse\outputon
    \else\ifx\answer@AmS\n@AmS\retry@AmSfalse\outputoff
     \else\ifx\answer@AmS\n@@AmS\retry@AmSfalse\outputoff
      \else \retry@AmStrue\fi\fi\fi\fi
  \ifretry@AmS\message{Type y or n, follow answer by return: }\repeat}

\def\outputoff{\global\output{\setbox0=\box255 \deadcycles=0}}

\def\outputon{\global\output{\output@AmS}}
\outputon

\catcode`\@=13
